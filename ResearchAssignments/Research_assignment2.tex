\documentclass[linenumbers, preprint, times]{aastex631}

\newcommand{\vdag}{(v)^\dagger}
\newcommand\aastex{AAS\TeX}
\newcommand\latex{La\TeX}

\usepackage{graphicx}
\usepackage{fancyvrb}
\usepackage{listings}
\usepackage{caption}
\begin{document}

\nolinenumbers
\title{Research Paper Proposal
\footnote{Released on March, 16th, 2023}}

\author{Suresh Surya}
\affiliation{DEPARTMENT OF ASTRONOMY, University of Arizona.
933 North Cherry Avenue, 
Tucson, AZ 85721-0065} 
\section{Introduction} \label{sec:style}
This Paper is my Research Proposal for ASTR 400B. My chosen topic is
on the Stellar disk particle distribution/morphology of the Milky Way and the M31 galaxies
Stellar Major Merger Remnant. That is the distribution of stellar particles in the disk of the
remnant of the merger between The Milky Way Galaxy and M31 (Andromeda) Galaxy.\par
This topic is closely related to Galactic evolution. Galactic mergers have been considered to drive the evolution of galaxies.  The Merger between M31 and the Milky Way would lead to a remnant being formed, this remnant would have a changing composition and star formation rates. 
With time the remnant's star formation will slow down leading to older stars dominating the composition. The life cycle of
the remnant is similar to galactic evolution. It is important to study the Stellar disk particle distribution of merger
remnants to improve on our current understanding of galactic evolution. It is also important to understand whether there is 
any link between galaxy mergers and morphology. Will 2 barred-spiral galaxies of similar masses collide to produce an elliptical? A spiral? irregular? If there is a link between galaxy evolution and stellar populations, will the
remnant have a composition similar to that of a young galaxy? Also, considering Andromeda has more baryonic matter than the Milky Way, would the remnant resemble Andromeda, the Milky Way, or neither? These are important topics in understanding
galactic evolution.\par
Multiple simulations of the MW and M31 merger have been done to date. There are many results, for example, according to \citet{Toomre_1972}, spiral galaxy mergers should likely lead to elliptical remnants. This is one of the more common theories surrounding galaxy mergers. Another theory posited by \citet{Querejeta_2015} is the formation of an S0 galaxy after a galactic collision. They claim that there is a difference between late-type spirals like Sb, Sc, etc. compared to S0 galaxies in characteristics like angular momentum and concentration and that a likely reason for this is that some S0's are formed as merger remnants. According to \citet{Pearson_2019}, the role of mergers in affecting star formation rates is still contested, although mergers have been found to undergo bursts of star formation, this is only found in 10-20\% of mergers. They claim that higher mass galaxies $(> 10^{10.7}M_\odot)$ are more often gas-poor and "dry" mergers sometimes suppress star formation. Considering both the Milky Way and M31 are this massive, there is a possibility that the merger will be dry. Considering this, in a few million years, the remnant would most likely be composed of older stars.\par
The open questions are: Will the galactic merger be dry? Considering that both Milky Way and Andromeda are high-mass galaxies, will the population of stars in the remnant resemble older galaxies?
What will the classification of the remnant be? There are several possibilities and depending on the simulation properties can lead to different classes. However, Elliptical galaxies are more common merger remnants.
Will the remnant be more similar to Andromeda or the Milky Way or neither? Considering Andromeda is more massive, What would be the stellar distribution of the remnant? What about its stellar density profile? Finally, what about the structure of the remnant? Will it have a disk, would it have spiral arms or a ring of gas as some galaxies do? 
\begin{figure}
    \centering
    \includegraphics[width=10cm]{aa38674-20-fig3.jpg}
    \caption{Stellar density profile of Milky way, Andromeda and "Milkomeda" which is the merger remnant the authors simulated. Image by \citet{Schiavi_2020} }
    \label{fig:galaxy}
\end{figure}
\par I found the Figure 1 taken from the paper "Future merger of the Milky Way with the Andromeda galaxy and the fate of their supermassive black holes" by \citet{Schiavi_2020} to be particularly helpful in motivating the research paper, as it shows for example the stellar density profiles of Milky Way, M31 and the remnant. 
\section{Proposal} \label{sec:style}
\subsection{Questions to be addressed}
In the Research Paper, the question(s) to be investigated are:\par
What is the final stellar density profile for the combined system ? Is it well fit by
a Sérsic profile? Does it agree with predictions for elliptical galaxies?\par
And/or \par
What is the distribution of stellar particles from M31 vs the MW? Are the profiles
different? \par
\subsection{Approach to Simulation}
In order to answer these questions, simulation data is required. Firstly, each galaxy has to be simulated before and after the merger. We must know the positions of particles in Milky Way and M31 before and after the merger. Using text files that contain Particle data we can find out various properties of MW and M31. We can assume that Milky way and M31 are gas poor and that the merger is "dry" so we don't have to account for gas. The text files that contain particle data vary based on time. We can then estimate the time at which the merger takes place. As text files that contain particle data were used, we can trace back which particles belong to the Milky Way and which belong to M31. Now in order to find the particle data of the remnant, we have to combine the two galaxies. The way to do this is to use the text files at some point t to make two arrays, one for Milky Way and one for M31. This can be done using \verb+Readfile+, which contains a variable \verb+data+ that holds the array.
Then we make an array with the particle data of the remnant. To do this, we concatenate both M31 and MW arrays. In this case, we are looking for the Sérsic profile, so we only need the bulge particle type. However, if the remnant is elliptical then the disk must also be included. Once we have the array that contains particle data for the remnant, we can compute stellar density profiles of the combined system that is the remnant. In order to find this there
Once found we plot the Stellar density profiles in terms of column density and radius. We then compute the surface brightness of the remnant. Plotting the Surface brightness vs radius, we try to fit a Sérsic profile to the plot. The code to find the stellar density profiles and finding the Sérsic profiles is provided in class. In order to find the stellar density profile we use components developed before, such as \begin{verbatim} CenterOfMass, MassProfile, GalaxyMass \end{verbatim} First we use the class \verb+MassProfile+, in order to find the Mass profile of the Bulge. Instead of finding the total mass of the bulge, we have to find the mass in shells. Assume a certain part of the bulge with mass $\Delta M$ and with radius $\Delta r$ centered on the center of the bulge. We compute the density, which has the formula: \[\rho =  \frac{M}{\frac{4\pi r^3}{3}} \approx\frac{M}{r^3}\] when using $\Delta M$ and $\Delta r$:
\[\rho = \frac{\Delta M}{\Delta r^3}\] Since $\rho$ varies based on r, we will be able to make a function that finds $\rho$ based on different values of r, until $\Delta r = R$. This is done by using \verb+MassProfile+.
Then we plot $\rho\;vs\;r$ where r goes from 0 to R. R, is the radius of the remnant. Then we calculate the effective radius, that is the half mass radius for the bulge. Once computed, we need to define a function that estimates the Sérsic profile, this Sérsic profile can be calculated using the following formula: \[
I(r) = I_e exp{[-7.67 ( (r/R_e)^{1/n} - 1)]}
\] and \[L = 7.2 I_e \pi R_e^2\] Once calculated, we can plot the Sérsic profile and try to fit it with some value of n.
\subsection{Figure to illustrate Methodology}
\par
Figure 2 illustrates the goal of this research project. We want to find the Surface brightness of the merger remnant, and see if we can fit a Sérsic profile to it. Although the surface brightness of the remnant may not be similar to the one given in the figure, it illustrates the methodology of the process.
\begin{center}
\includegraphics[width = 10cm]{download.png}%
\captionof{figure}{Sérsic profile fitted to log normalized Surface brightness as a function of radius in Kpc.}\label{labelname}%
\end{center}
\par
\subsection{Hypothesis}
The merger between M31 and MW might result in some interesting conclusions. A possibility is that the remnant is more similar to an elliptical galaxy that is elongated, maybe E5 or higher. We might find a stellar density profile more similar to that of M31 as it is more massive and larger in size than the milky way, My theory is that as shown in Figure 1, the stellar density profile of the remnant would be higher than that of the Milky Way and Andromeda. I personally don't think that the merger would produce an S0 galaxy. Another possibility is that the Sérsic profile of the remnant will be different from that of the MW and M31. And that there might be an even distribution of stellar particles from M31 and MW.

\nolinenumbers
\bibliography{references}{}
\bibliographystyle{aasjournal}
\end{document}
