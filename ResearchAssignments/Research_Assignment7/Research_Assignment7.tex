\documentclass[preprint2, times]{aastex631}
\newcommand{\vdag}{(v)^\dagger}
\newcommand\aastex{AAS\TeX}
\newcommand\latex{La\TeX}
\usepackage{graphicx}
\usepackage{fancyvrb}
\usepackage{listings}
\usepackage{caption}
\begin{document}

\title{Stellar Density Profiles in the MW-M31 Merger Remnant
\footnote{Released on March, 16th, 2023}}

\author{Suresh Surya}
\affiliation{DEPARTMENT OF ASTRONOMY, University of Arizona.
933 North Cherry Avenue, 
Tucson, AZ 85721-0065} 
\begin{abstract}
The goal of this research paper is to analyze the merger remnant of MW and M31 galaxies. We can understand more about the evolution of galaxies by analyzing merger remnants. The question that we are looking to answer is what is the stellar density profile of the remnant. This is important because we can learn more about the structure of a galaxy by analyzing the stellar density profile. We found that the remnant exhibits characteristics of both an elliptical and a spiral galaxy. We also found that the stellar density profile of the remnant is more similar to that of M31 than the Milky Way. This suggests that potentially a merger between the Milky Way and M31 may form an elliptical galaxy, that is more massive than both galaxies.
\end{abstract}
\keywords{Major Merger: A merger between two massive galaxies --- Particle distribution: The distribution of particles throught a galaxy --- Merger Remnant: The remnant of a galactic merger --- Stellar Disk: The disk component of a galaxy that contains most of the stars. --- Dry Merger: A merger that surpresses star formation --- Spiral Galaxy: A galaxy with a well-defined disk and spiral arms that extend from the center of the galaxy outwards}

\section{Introduction} \label{sec:style}
This paper presents a study of the stellar disk particle distribution and morphology of the remnant of the major merger between the Milky Way and M31 (Andromeda) galaxies. The goal of this research is to analyze the distribution and morphology of the stellar particles in the disk of the remnant, and to compare it with theoretical predictions and observations of other similar galaxies. The research was conducted through a combination of numerical simulations and data analysis of observational data from various sources.\par
What is a Galaxy? According to \citet{Willman_2012}, A galaxy is defined as a gravitationally bound system of stars, gas, dust, and dark matter that typically has a flattened morphology and is organized hierarchically in a clustering of systems that span a wide range of masses. 
This topic is closely related to galactic evolution, as galactic mergers are considered to drive the evolution of galaxies. Galactic evolution refers to the processes that lead to the formation, growth, and changing appearance of galaxies over time.
The merger between Andromeda and the Milky Way would lead to the formation of a remnant with a changing composition and star formation rates.  
Over time, the remnant's star formation will slow down, leading to older stars dominating its composition. The life cycle of the remnant is similar to galactic evolution. Therefore, studying the stellar disk particle distribution of merger remnants is important to improve our current understanding of galactic evolution. It is also important to understand whether there is 
any link between galaxy mergers and morphology. Will two barred-spiral galaxies of similar masses collide to produce an elliptical, spiral, or irregular galaxy? If there is a link between galaxy evolution and stellar populations, will the remnant have a composition similar to that of a young galaxy? Considering Andromeda has more baryonic matter than the Milky Way, would the remnant resemble Andromeda, the Milky Way, or neither? These are important topics in understanding
galactic evolution.\par
Multiple simulations of the MW and M31 merger have been done to date resulting in various findings. For example, according to \citet{Toomre_1972}, spiral galaxy mergers should likely lead to elliptical remnants. This is one of the more common theories surrounding galaxy mergers. Another theory posited by \citet{Querejeta_2015} is the formation of an S0 galaxy after a galactic collision. They claim that there are differences between late-type spirals like Sb, Sc, etc. compared to S0 galaxies in characteristics like angular momentum and concentration. This difference may be due to some S0 galaxies forming as merger remnants. According to \citet{Pearson_2019}, the role of mergers in affecting star formation rates is still contested, although mergers have been found to undergo bursts of star formation, this is only found in 10-20\% of mergers. They claim that higher mass galaxies $(> 10^{10.7}M_\odot)$ are more often gas-poor and "dry" mergers sometimes suppress star formation. Considering both the Milky Way and M31 are this massive, there is a possibility that the merger will be dry.  If this is the case, in a few million years, the remnant would most likely be composed of older stars.\par
The open questions are: Will the galactic merger be dry? Considering that both Milky Way and Andromeda are high-mass galaxies, will the population of stars in the remnant resemble older galaxies?
What will the classification of the remnant be? There are several possibilities that can lead to different classifications depending on the simulation properties, but elliptical galaxies are more common merger remnants. 
Will the remnant be more similar to Andromeda or the Milky Way or neither? Considering Andromeda is more massive, what would be the stellar distribution and stellar density profile of the remnant? Finally, will the remnant exhibit a disk-like morphology, as seen in both the Milky Way and Andromeda, or will it be more irregular in shape?
\begin{figure}
    \centering
    \plotone{aa38674-20-fig3.jpg}
    \caption{Stellar density profile of Milky way, Andromeda and "Milkomeda" which is the merger remnant the authors simulated. Image by \citet{Schiavi_2020} }
    \label{fig:galaxy}
\end{figure}
\vspace{5 mm}
\section{This Project}\label{sec:style}\
In this paper, we will examine the Merger Remnant of the Milky Way and M31 galaxies. Our goal is to use N-body simulations to model the stellar density profile of the Major Merger Remnant. Then, we plan to see if we can fit a Sersic profile to this stellar density profile. Based on the sersic profile our plan is to try to make some conclusions about the remnant, and hopefully answer whether our data agrees with predictions for elliptical galaxies. We can also compare our results with that of the Milky Way and M31 as like that of figure 1.
\par
The results of this project can address the open questions Will the remnant be more similar to Andromeda or the Milky Way or neither? and is the remnant an elliptical or spiral galaxy? For the first question, we can plot the stellar density profiles of the Milky Way and M31 along with the remnant and see if there are any similarities. For the second question, we can compare the stellar density profile and Sersic profile with that of an elliptical galaxy and see if there are any similarities.
\par
Understanding the properties of the merger remnant is an important problem to solve for our understanding of galaxy evolution. It can provide insights into the formation and evolution of galaxies through major mergers, which is thought to be a common process in the hierarchical growth of galaxies. By answering whether the remnant will be more similar to Andromeda or the Milky Way, we can gain insight into the factors that determine the properties of the remnant after a major merger, such as the mass ratio and orbital parameters of the progenitor galaxies. Additionally, determining whether the remnant is an elliptical or spiral galaxy can shed light on the role of major mergers in driving galaxy morphology transformations.

Our study can help to address these open questions by using N-body simulations to model the stellar density profile of the major merger remnant, and fitting a Sersic profile to this profile. By comparing the results with those of the Milky Way and M31 and with those of elliptical galaxies, we can draw conclusions about the properties of the merger remnant and how they relate to the properties of the progenitor galaxies. This can provide important information for testing models of galaxy formation and evolution, and for improving our understanding of the processes that shape the properties of galaxies over cosmic time. 
\section{Methodology}\label{sec:style}\
To achieve our research goals, we are going to be using simulations to model MW and M31. We will use N-body simulations to study the interaction between the Milky Way and M31 galaxies. Specifically, we will use the simulations described in the paper "Modeling the Collision Between the Milky Way and Andromeda from the View of Andromeda" by van der Marel and Besla (2012). N-body simulations are a type of simulation in which the motion of particles is simulated using Newton's laws of motion. In the context of galaxy simulations, N-body simulations are used to simulate the motion of stars and dark matter particles in a galaxy.
\par
In figure 2, is the stellar density profie of M31, from 0.1 to 50 kpc. This shows the distribution of particles in the disk and bulge of M31. The units of the stellar density profile are given in $(10^{10}M_\odot / kpc^2)$. And the plot is a log-log plot.Hence a value of $10^1$ on the y-axis would mean $10^{11}M_\odot / kpc^2)$. This is the ideal graph we would like to obtain from our data, we would want to measure the stellar density profile of the new merged galaxy, and possibly compare it to that of MW and M31.
\par
Our approach is to use the simulation data to compute the surface density profile of the M31 galaxy. The simulation data is provided in class and is in the form of text files that have data about particles in the MW and M31. First we define code to calculate the center of mass of the galaxy, the general equation/method used to do this is using weighted averaging:
\begin{equation}
\mathrm{COM} = \frac{\sum\limits_{i=1}^N m_i \cdot \vec{x}i}{\sum\limits_{i=1}^N m_i}
\end{equation}
\begin{figure}
    \centering
    \plotone{download (1).png}
    \caption{Stellar density profile of M31 from Lab 6}
    \label{fig:galaxy}
\end{figure}
Here, $x_i$ is the $i^{th}$ component of the vector quantity and $m_i$ is the mass of the $i^{th}$ particle. This is done for x,y and z components to determine the center of mass. Now, to simulate the merger we combine the particle data for Milky Way and M31. We calculate the relative position of each particle to their galaxy's COM. We then Compute the cylindrical coordinates (radius and azimuthal angle) of the particles in the combined system.
Define a range of radii and use this to create a mask to select particles within each annulus.
Calculate the enclosed masses for each annulus by summing the masses of the particles within it.
Use the differences between adjacent enclosed masses to obtain the mass in each annulus.
Calculate the surface density in each annulus by dividing the mass by the area of the annulus.
\begin{equation}
\mathrm{\Sigma} = \frac{m_{annuli}}{\pi(r_{out}^2-r_{in}^2)}
\end{equation}
Where $\Sigma$ is the Surface density profile of the annulus, $m_{annuli}$ is the mass in the annulus, and $r_{in}$ and $r_{out}$ are the inner and outer radii of the annulus, respectively. Now, we turn Stellar density profile into a Surface Brightness profile. In order to do this, we determine the effective radius and the sersic profile (\citet{Sersic_1963}), using the surface density profile. We make the assumption that M/L = 1, therefore the effective radius is equal to the half light radius.
\par
The plots needed for this project are: A plot of surface density profile of the Merger remnant, along with that of the MW and M31. And the surface brightness plot of the merger remnant at snapshots 0, 0+n, n+n, 2n+n etc. We need a high resolution plot of the surface brightness profile along with a fitted sersic profile to determine if there are any similarities to an elliptical galaxy. Finally we need a cross-sectional diagram of the galaxy remnant, to illustrate the shape of the remnant.
\par
We hypothesize that the merger between the MW and M31, might result in a galaxy that is more elliptical than M31 and MW, but may not be an elliptical galaxy in terms of classification. Another possibility is that the remnant is a spiral galaxy similar to M31, with some properties of the Milky Way.
\section{Results}
The first figure (Figure 3) is a figure that shows the stellar density profile of the merger remnant of the Milky Way Galaxy and M31. The plot follows the same format as shown in figure 2, with the units of the y-axis being in $(10^{10}M_\odot / kpc^2)$ and the x-axis being in Kpc. It is a log-log plot. For this plot, we used high-resolution data for computing the stellar density profile of the merger remnant. It was made from code that simulates the galactic merger and plots the surface density profile at snapshot 000.
\begin{figure}
    \centering
    \plotone{MW_M31_000.png}
    \caption{Stellar density profile of M31 and MW merger remnant at snapshot 000.}
    \label{fig:galaxy}
\end{figure}
\par
\begin{figure}
    \centering
    \plotone{MW+M31+Remnant.png}
    \caption{Stellar density profile of M31 and MW merger remnant at snapshot 000 in comparison with MW and M31 seperately.}
    \label{fig:galaxy}
\end{figure}
\begin{figure}
    \centering
    \plotone{Lab7_EdgeOn_Density.png}
    \caption{MW+M31 merger remnant Face On particle distribution}
    \label{fig:galaxy}
\end{figure}
\begin{figure}
    \centering
    \plotone{Lab7_FaceOn_Density.png}
    \caption{MW+M31 merger remnant Face On particle distribution}
    \label{fig:galaxy}
\end{figure}
The next figure (Figure 4) shows the stellar density profile of the merger remnant in comparison to that of the Milky Way Galaxy and M31. The plot follows the same format as shown in figure 2, with the units of the y-axis being in $(10^{10}M_\odot / kpc^2)$ and the x-axis being in Kpc. The blue line dictates the simulated bulge of the simulated merger remnant, the orange line is the M31's stellar density profile and the green line is MW's stellar density profile. It was made from code that calculates the stellar density profile for the MW, Andromeda and the merger remnant and plots them together.
\par
The next two figures show the shape of the galaxy when modeled on to a 2D plot. Figure 5 shows the MW+M31 merger 
remnant's distribution of particles in a face-on view. Whereas figure 6 shows the remnant's distribution of particles in an edge-on view. The colored bar represents the particle density of the merger remnant, This code is taken from lab 7, 
and is modified to use the input values of the MW+M31 merger remnant. 
\section{Discussion}
The stellar density profile of the merger remnant between the Milky Way (MW) and Andromeda (M31) galaxies provides valuable insights into the formation and evolution of galactic mergers. One of the key observations from our research is that the density profile of the merger remnant is more similar to that of M31, suggesting that it has inherited more characteristics from M31 than from the MW. This similarity in the density profile could be attributed to the relative mass of the two galaxies, with the remnant likely being more massive due to the dominance of M31.

Furthermore, the density profile analysis indicates that the merger remnant is either an elliptical galaxy or possesses a large bulge. This conclusion is based on the relatively flat profile observed up to approximately 5 parsecs. Elliptical galaxies are known for their smooth, featureless density distributions, while galaxies with a prominent bulge exhibit a similar behavior in their density profiles. The absence of significant structures or fluctuations in the density profile suggests a lack of pronounced spiral arms or disk-like structures in the remnant.

When visualizing the merger remnant on a two-dimensional plot, it becomes evident that the remnant exhibits tiny arms but is predominantly circular up to a certain point. This finding aligns with the characteristics of elliptical galaxies, which often lack well-defined spiral arms but can exhibit some degree of asymmetry or faint features. The circularity observed in the remnant's 2D projection indicates a more spheroidal shape rather than a flattened disk-like structure.

In addition to the morphological aspects, the size of the merger remnant stands out as a significant feature. It is notably larger than both the MW and M31, emphasizing the substantial growth and expansion resulting from the merger event. The increased size of the remnant could be a consequence of the combined mass of the merging galaxies, leading to the redistribution and stretching of stellar material.

An intriguing observation arises when examining the merger remnant from an edge-on view. The presence of particles significantly above the "plane" of the galaxy implies a departure from a purely disk-like morphology. This observation provides further support for the possibility of the remnant being an elliptical galaxy, as such galaxies often exhibit a three-dimensional distribution of stars that deviates from a thin disk.

In summary, the stellar density profile analysis of the merger remnant between the MW and M31 suggests that the remnant bears a closer resemblance to M31, likely due to its higher mass. The flat profile up to approximately 5 parsecs indicates the presence of an elliptical galaxy or a significant bulge. The visualization of the remnant on a 2D plot reveals faint arms and a predominantly circular shape, consistent with elliptical galaxies. The larger size of the remnant compared to the progenitor galaxies highlights the transformative nature of galactic mergers. Lastly, the edge-on view provides hints of a three-dimensional distribution, supporting the possibility of the remnant being an elliptical galaxy. These findings contribute to our understanding of galaxy mergers and the resulting structures that emerge from such events.
\section{Conclusion}
This research paper examines the merger remnant of the Milky Way and M31 galaxies to understand galactic evolution. Specifically, we analyze the stellar density profile of the remnant, which provides insights into galaxy structure. Our findings reveal a hybrid galaxy with characteristics of both elliptical and spiral galaxies, leaning towards M31's profile. This suggests a potential formation of a more massive elliptical galaxy resulting from the merger.
\par
A key finding is the coexistence of elliptical and spiral galaxy features in the merger remnant, aligning with our hypothesis. This supports the notion that mergers lead to complex structures. These results strengthen our research approach and understanding of galaxy evolution through mergers.
\par
Future directions include detailed investigations of substructures within the remnant, exploring additional parameters (e.g., gas content, star formation rate), and improving analysis code for accuracy and efficiency. Incorporating larger datasets and studying other mergers would validate our findings and broaden our understanding of galactic evolution.
In conclusion, our research highlights a hybrid merger remnant, combining elliptical and spiral galaxy characteristics. These findings support our hypothesis and indicate the potential formation of a more massive elliptical galaxy from the Milky Way-M31 merger. Further studies and improved analysis techniques will deepen our understanding of galaxy mergers and their role in shaping the universe.
\section{Acknowledgements}:
I would like to express my gratitude to the following individuals and organizations for their contributions to this research:

The creators and developers of Matplotlib and NumPy, Astropy and Scipy for providing the essential tools and libraries that enabled the visualization and analysis of the data presented in this study.

My professor, Dr. Gurtina Besla for providing the particle data and code necessary for the analysis.

I acknowledge the invaluable contributions of the aforementioned individuals and organizations, which greatly enhanced the quality and depth of this research.
\bibliography{references}{}
\bibliographystyle{aasjournal}


%% This command is needed to show the entire author+affiliation list when
%% the collaboration and author truncation commands are used.  It has to
%% go at the end of the manuscript.
%\allauthors

%% Include this line if you are using the \added, \replaced, \deleted
%% commands to see a summary list of all changes at the end of the article.
%\listofchanges

\end{document}

% End of file `sample631.tex'.
